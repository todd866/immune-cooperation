\documentclass[11pt]{letter}
\usepackage[margin=1in]{geometry}
\usepackage{hyperref}
\usepackage{parskip}

\signature{Ian Todd\\Sydney Medical School\\University of Sydney\\itod2305@uni.sydney.edu.au}
\address{Ian Todd\\Sydney Medical School\\University of Sydney\\Sydney, NSW, Australia}

\begin{document}

\begin{letter}{Professor Abir Igamberdiev\\Editor-in-Chief\\BioSystems}

\opening{Dear Professor Igamberdiev,}

I am pleased to submit ``The Physics of Immune Cooperation: Dimensional Surveillance and Attractor Enforcement in Multicellular Systems'' for consideration in \emph{BioSystems}.

This manuscript proposes a unified theoretical framework that resolves the ``combinatorial explosion'' paradox in immunology by reframing immune recognition as a problem of \textbf{dimensional surveillance}. Drawing on attractor dynamics, costly signalling theory, and the bioelectric frameworks of Levin et al., I argue that the immune system does not merely recognize molecular patterns but actively measures the dynamical complexity ($D_{\text{eff}}$) of target cells.

Cancer cells, virally-infected cells, and senescent cells share a common signature: reduced effective dimensionality, reflecting escape from high-dimensional organismal attractors into simpler replicative or dysfunctional states. Immune receptor interactions (CD molecules, MHC-TCR coupling) function as \textbf{synchronization probes} that couple to target cells and measure their dynamical complexity. This reframing unifies cancer immune evasion, T-cell exhaustion, autoimmunity, and chronic inflammation under a single dynamical principle.

\textbf{Key Contributions:}
\begin{enumerate}
    \item \textbf{Dynamical Friction and Exhaustion:} I derive T-cell exhaustion not as a functional defect but as a thermodynamic consequence of ``dynamical friction''---the metabolic cost of coupling a high-dimensional sensor to a low-dimensional target. Exhaustion time scales as $t_{\text{exhaust}} \sim E_0 / k(\Delta D)^2$.

    \item \textbf{Sensor Collapse and Inflamm-aging:} When the immune sensor itself loses dimensionality, distinct target states alias onto the same low-dimensional projection, producing either indiscriminate attack (autoimmunity) or indiscriminate tolerance (exhaustion)---explaining the paradox of inflamm-aging.

    \item \textbf{Empirical Validation:} Crucially, I test this theory against single-cell RNA sequencing data from melanoma patients (GSE120575). T-cells from immunotherapy responders exhibit \textbf{2.3-fold higher effective dimensionality} (participation ratio $\approx 28$) compared to non-responders ($\approx 12$).
\end{enumerate}

\textbf{The Dimensionality-Entropy Dissociation:} Our analysis reveals a critical theoretical distinction. While responders have higher \textit{dimensional structure} ($D_{\text{eff}}$), they do not have higher \textit{transcriptomic entropy}---in fact, entropy is slightly higher in non-responders. This empirically distinguishes \textbf{coherent complexity} (health) from \textbf{incoherent noise} (disease), validating the core premise that the immune system targets loss of functional degrees of freedom, not merely thermodynamic disorder.

This submission continues a research program on dimensional constraints in biology:
\begin{itemize}
    \item Todd (2025a): ``The limits of falsifiability'' (DOI: 10.1016/j.biosystems.2025.105608)
    \item Todd (2025b): ``Timing inaccessibility and the projection bound'' (DOI: 10.1016/j.biosystems.2025.105632)
\end{itemize}

The manuscript engages substantially with Cohen et al.\ (2022, \emph{Nature Aging}) on complex systems approaches to aging, and Levin (2021, \emph{Cell}) on bioelectric signaling and morphogenetic fields. All simulation code and analysis scripts are available at \url{https://github.com/todd866/immune-cooperation}.

This manuscript is original and not under consideration for publication elsewhere.

\closing{Thank you for your consideration.}

\end{letter}
\end{document}
